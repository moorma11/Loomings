\documentclass{article}

\author{Chad Moorman}
\title{Loomings}
\begin{document}
\maketitle


\section{2022-01-27:Call Me Ishmael}
I created a radiation boundary condition in the FEM code and verified that it is being assigned correctly in geom.f90.  I also wrote mult routines for the first five block on the LHS for WE-FEB.  Finally, I reread the div conforming basis chapter in Peterson to better understand RWG basis functions, as I will use them to represent surface currents on the truncation boundary in the DFIE framework.  Peterson writes the RWG basis set in terms of simplex coordinates, but the IEEE slides Shanker gave me express RWG in terms of cartesian space.  I need to compute the Jacobian by hand to understand this mapping.
\section{2022-01-27:}
Todays FEM meeting, zane showed me how impressed surface currents are defined in the FEM code.  
\section{2022-01-31:Resuming in person classes}
Resuming in person classes today.  Going back to the office to see if anyone stole my desk.
\section{2022-01-31:Resuming in person classes}
We're resuming in person classes
\begin{itemize}
    \item Compute by hand the N cross N block to verify that it is correct
    \item CMSE 821 Hw1 Due Friday
    \item Annual Report due Sunday
    \item ECE 836 Hw1 due Sunday
    \item Fix FEM bug
\end{itemize}
\end{document}
